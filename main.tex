\documentclass[a4paper,14pt,oneside,openany]{memoir}

%%% Задаем поля, отступы и межстрочный интервал %%%

\usepackage[left=30mm, right=15mm, top=20mm, bottom=20mm]{geometry} % Пакет geometry с аргументами для определения полей
\pagestyle{plain} % Убираем стандарные для данного класса верхние колонтитулы с заголовком текущей главы, оставляем только номер страницы снизу по центру
\parindent=1.25cm % Абзацный отступ 1.25 см, приблизительно равно пяти знакам, как по ГОСТ
\usepackage{indentfirst} % Добавляем отступ к первому абзацу
%\linespread{1.3} % Межстрочный интервал (наиболее близко к вордовскому полуторному) - тут вместо этого используется команда OnehalfSpacing*

%%% Задаем языковые параметры и шрифт %%%

\usepackage[english, russian]{babel}                % Настройки для русского языка как основного в тексте
\babelfont{rm}{Times New Roman}                     % TMR в качестве базового roman-щрифта

%%% Задаем стиль заголовков и подзаголовков в тексте %%%

\setsecnumdepth{subsection} % Номера разделов считать до третьего уровня включительно, т.е. нумеруются только главы, секции, подсекции
\renewcommand*{\chapterheadstart}{} % Переопределяем команду, задающую отступ над заголовком, чтобы отступа не было
\renewcommand*{\printchaptername}{} % Переопределяем команду, печатающую слово "Глава", чтобы оно не печалось
%\renewcommand*{\printchapternum}{} % То же самое для номера главы - тут не надо, номер главы оставляем
\renewcommand*{\chapnumfont}{\normalfont\bfseries} % Меняем стиль шрифта для номера главы: нормальный размер, полужирный
\renewcommand*{\afterchapternum}{\hspace{1em}} % Меняем разделитель между номером главы и названием
\renewcommand*{\printchaptertitle}{\normalfont\bfseries\centering\MakeUppercase} % Меняем стиль написания для заголовка главы: нормальный размер, полужирный, центрированный, заглавными буквами
\setbeforesecskip{20pt} % Задаем отступ перед заголовком секции
\setaftersecskip{20pt} % Ставим такой же отступ после заголовка секции
\setsecheadstyle{\raggedright\normalfont\bfseries} % Меняем стиль написания для заголовка секции: выравнивание по правому краю без переносов, нормальный размер, полужирный
\setbeforesubsecskip{20pt} % Задаем отступ перед заголовком подсекции
\setaftersubsecskip{20pt} % Ставим такой же отступ после заголовка подсекции
\setsubsecheadstyle{\raggedright\normalfont\bfseries}  % Меняем стиль написания для заголовка подсекции: выравнивание по правому краю без переносов, нормальный размер, полужирный

%%% Задаем параметры оглавления %%%

\addto\captionsrussian{\renewcommand\contentsname{Содержание}} % Меняем слово "Оглавление" на "Содержание"
\setrmarg{2.55em plus1fil} % Запрещаем переносы слов в оглавлении
%\setlength{\cftbeforechapterskip}{0pt} % Эта команда убирает интервал между заголовками глав - тут не надо, так красивее смотрится
\renewcommand{\aftertoctitle}{\afterchaptertitle \vspace{-\cftbeforechapterskip}} % Делаем отступ между словом "Содержание" и первой строкой таким же, как у заголовков глав
%\renewcommand*{\chapternumberline}[1]{} % Делаем так, чтобы номер главы не печатался - тут не надо
\renewcommand*{\cftchapternumwidth}{1.5em} % Ставим подходящий по размеру разделитель между номером главы и самим заголовком
\renewcommand*{\cftchapterfont}{\normalfont\MakeUppercase} % Названия глав обычным шрифтом заглавными буквами
\renewcommand*{\cftchapterpagefont}{\normalfont} % Номера страниц обычным шрифтом
\renewcommand*{\cftchapterdotsep}{\cftdotsep} % Делаем точки до номера страницы после названий глав
\renewcommand*{\cftdotsep}{1} % Задаем расстояние между точками
\renewcommand*{\cftchapterleader}{\cftdotfill{\cftchapterdotsep}} % Делаем точки стандартной формы (по умолчанию они "жирные")
\maxtocdepth{subsection} % В оглавление попадают только разделы первыхтрех уровней: главы, секции и подсекции

%%% Выравнивание и переносы %%%

%% http://tex.stackexchange.com/questions/241343/what-is-the-meaning-of-fussy-sloppy-emergencystretch-tolerance-hbadness
%% http://www.latex-community.org/forum/viewtopic.php?p=70342#p70342
\tolerance 1414
\hbadness 1414
\emergencystretch 1.5em                             % В случае проблем регулировать в первую очередь
\hfuzz 0.3pt
\vfuzz \hfuzz
%\dbottom
%\sloppy                                            % Избавляемся от переполнений
\clubpenalty=10000                                  % Запрещаем разрыв страницы после первой строки абзаца
\widowpenalty=10000                                 % Запрещаем разрыв страницы после последней строки абзаца
\brokenpenalty=4991                                 % Ограничение на разрыв страницы, если строка заканчивается переносом

%%% Объясняем компилятору, какие буквы русского алфавита можно использовать в перечислениях (подрисунках и нумерованных списках) %%%
%%% По ГОСТ нельзя использовать буквы ё, з, й, о, ч, ь, ы, ъ %%%
%%% Здесь также переопределены заглавные буквы, хотя в принципе они в документе не используются %%%

\makeatletter
    \def\russian@Alph#1{\ifcase#1\or
       А\or Б\or В\or Г\or Д\or Е\or Ж\or
       И\or К\or Л\or М\or Н\or
       П\or Р\or С\or Т\or У\or Ф\or Х\or
       Ц\or Ш\or Щ\or Э\or Ю\or Я\else\xpg@ill@value{#1}{russian@Alph}\fi}
    \def\russian@alph#1{\ifcase#1\or
       а\or б\or в\or г\or д\or е\or ж\or
       и\or к\or л\or м\or н\or
       п\or р\or с\or т\or у\or ф\or х\or
       ц\or ш\or щ\or э\or ю\or я\else\xpg@ill@value{#1}{russian@alph}\fi}
\makeatother

%%% Задаем параметры оформления рисунков и таблиц %%%

\usepackage{graphicx, caption, subcaption} % Подгружаем пакеты для работы с графикой и настройки подписей
\graphicspath{{images/}} % Определяем папку с рисунками
\captionsetup[figure]{font=small, width=\textwidth, name=Рисунок, justification=centering} % Задаем параметры подписей к рисункам: маленький шрифт (в данном случае 12pt), ширина равна ширине текста, полнотекстовая надпись "Рисунок", выравнивание по центру
\captionsetup[subfigure]{font=small} % Индексы подрисунков а), б) и так далее тоже шрифтом 12pt (по умолчанию делает еще меньше)
\captionsetup[table]{singlelinecheck=false,font=small,width=\textwidth,justification=justified} % Задаем параметры подписей к таблицам: запрещаем переносы, маленький шрифт (в данном случае 12pt), ширина равна ширине текста, выравнивание по ширине
\captiondelim{ --- } % Разделителем между номером рисунка/таблицы и текстом в подписи является длинное тире
\setkeys{Gin}{width=\textwidth} % По умолчанию размер всех добавляемых рисунков будет подгоняться под ширину текста
\renewcommand{\thesubfigure}{\asbuk{subfigure}} % Нумерация подрисунков строчными буквами кириллицы
%\setlength{\abovecaptionskip}{0pt} % Отбивка над подписью - тут не меняем
%\setlength{\belowcaptionskip}{0pt} % Отбивка под подписью - тут не меняем
\usepackage[section]{placeins} % Объекты типа float (рисунки/таблицы) не вылезают за границы секциии, в которой они объявлены

%%% Задаем параметры ссылок и гиперссылок %%% 

\usepackage{hyperref}                               % Подгружаем нужный пакет
\hypersetup{
    colorlinks=true,                                % Все ссылки и гиперссылки цветные
    linktoc=all,                                    % В оглавлении ссылки подключатся для всех отображаемых уровней
    linktocpage=true,                               % Ссылка - только номер страницы, а не весь заголовок (так выглядит аккуратнее)
    linkcolor=black,                                  % Цвет ссылок и гиперссылок - красный
    citecolor=black                                   % Цвет цитировний - красный
}

%%% Настраиваем отображение списков %%%

\usepackage{enumitem}                               % Подгружаем пакет для гибкой настройки списков
\renewcommand*{\labelitemi}{\normalfont{--}}        % В ненумерованных списках для пунктов используем короткое тире
\makeatletter
    \AddEnumerateCounter{\asbuk}{\russian@alph}     % Объясняем пакету enumitem, как использовать asbuk
\makeatother
\renewcommand{\labelenumii}{\asbuk{enumii})}        % Кириллица для второго уровня нумерации
\renewcommand{\labelenumiii}{\arabic{enumiii})}     % Арабские цифры для третьего уровня нумерации
\setlist{noitemsep, leftmargin=*}                   % Убираем интервалы между пунками одного уровня в списке
\setlist[1]{labelindent=\parindent}                 % Отступ у пунктов списка равен абзацному отступу
\setlist[2]{leftmargin=\parindent}                  % Плюс еще один такой же отступ для следующего уровня
\setlist[3]{leftmargin=\parindent}                  % И еще один для третьего уровня

%%% Счетчики для нумерации объектов %%%

\counterwithout{figure}{chapter}                    % Сквозная нумерация рисунков по документу
\counterwithout{equation}{chapter}                  % Сквозная нумерация математических выражений по документу
\counterwithout{table}{chapter}                     % Сквозная нумерация таблиц по документу

%%% Реализация библиографии пакетами biblatex и biblatex-gost с использованием движка biber %%%

\usepackage{csquotes} % Пакет для оформления сложных блоков цитирования (biblatex рекомендует его подключать)
\usepackage[%
backend=biber,                                      % Движок
bibencoding=utf8,                                   % Кодировка bib-файла
sorting=none,                                       % Настройка сортировки списка литературы
style=gost-numeric,                                 % Стиль цитирования и библиографии по ГОСТ
language=auto,                                      % Язык для каждой библиографической записи задается отдельно
autolang=other,                                     % Поддержка многоязычной библиографии
sortcites=true,                                     % Если в квадратных скобках несколько ссылок, то отображаться будут отсортированно
movenames=false,                                    % Не перемещать имена, они всегда в начале библиографической записи
maxnames=5,                                         % Максимальное отображаемое число авторов
minnames=3,                                         % До скольки сокращать число авторов, если их больше максимума
doi=false,                                          % Не отображать ссылки на DOI
isbn=false,                                         % Не показывать ISBN, ISSN, ISRN
]{biblatex}[2016/09/17]
\DeclareDelimFormat{bibinitdelim}{}                 % Убираем пробел между инициалами (Иванов И.И. вместо Иванов И. И.)
\addbibresource{biba.bib}                           % Определяем файл с библиографией

%%% Скрипт, который автоматически подбирает язык (и, следовательно, формат) для каждой библиографической записи %%%
%%% Если в названии работы есть кириллица - меняем значение поля langid на russian %%%
%%% Все оставшиеся пустые места в поле langid заменяем на english %%%

\DeclareSourcemap{
  \maps[datatype=bibtex]{
    \map{
        \step[fieldsource=title, match=\regexp{^\P{Cyrillic}*\p{Cyrillic}.*}, final]
        \step[fieldset=langid, fieldvalue={russian}]
    }
    \map{
        \step[fieldset=langid, fieldvalue={english}]
    }
  }
}

%%% Прочие пакеты для расширения функционала %%%

\usepackage{longtable,ltcaption}                    % Длинные таблицы
\usepackage{multirow,makecell}                      % Улучшенное форматирование таблиц
\usepackage{booktabs}                               % Еще один пакет для красивых таблиц
\usepackage{soulutf8}                               % Поддержка переносоустойчивых подчёркиваний и зачёркиваний
\usepackage{icomma}                                 % Запятая в десятичных дробях
\usepackage{hyphenat}                               % Для красивых переносов
\usepackage{textcomp}                               % Поддержка "сложных" печатных символов типа значков иены, копирайта и т.д.
\usepackage[version=4]{mhchem}                      % Красивые химические уравнения
\usepackage{amsmath}                                % Усовершенствование отображения математических выражений 

%%% Вставляем по очереди все содержательные части документа %%%

\begin{document}

\thispagestyle{empty}

\begin{center}
    МИНИСТЕРСТВО ТРАНСПОРТА РОССИЙСКОЙ ФЕДЕРАЦИИ

    \vspace{20pt}

    Федеральное государственное автономное \\ образовательное учреждение высшего образования \\
    "<Российский университет транспорта"> \\

    \vspace{20pt}

    Кафедра <<Вычислительные системы, сети и \\ информационная безопасность>>
\end{center}

\vfill

\begin{center}
    Список вопросов \\  
    по лабораторным работам \\
    \textit{"<Низкоуровневые языки программирования">}

    \vspace{20pt}

    %по теме: \\
    %\uppercase{Длинное название темы моего безумного реферата, которое я даже выговорить не могу}
\end{center}

\vfill

    %\noindent Студент: \\
    %\textit{Группа № xxxx \hfill И.О. Фамилия}

    \vspace{20pt}

    \noindent Составил: \\
    \textit{соискатель учёной степени к.т.н. \hfill А. С. Филипченко}

\vfill

\begin{center}
    Москва "--- 2025 год
\end{center}                                     % Титульник

\newpage % Переходим на новую страницу
\setcounter{page}{2} % Начинаем считать номера страниц со второй
\OnehalfSpacing* % Задаем полуторный интервал текста (в титульнике одинарный, поэтому команда стоит после него)

\tableofcontents*                                   % Автособираемое оглавление

%\input{2_intro}                                     % Введение
\chapter{Пересылка и преобразование формата данных}
\begin{itemize}
\item Какими командами можно расширить однобайтное значение до двухбайтного?
\item Какими командами можно расширить двухбайтный код до 4-х байтного?
\item Какие два значения являются для процессора программной <<координатой>> байта в пространстве физической памяти при сегментной адресации?
\item Что означает <<косвенная адресация>> операнда в памяти? Какие варианты косвенной адресации возможны?
\item Перечислите имена 8-разрядных, 16-разрядных и 32-разрядных регистров процессора x386.
\end{itemize}

\section{Какими командами можно расширить однобайтное значение до двухбайтного?}

При работе с языком ассемблера TASM и процессорами архитектуры x86 одной из базовых задач является приведение чисел меньшей разрядности к большей. Это необходимо потому, что арифметические и логические операции в процессоре выполняются не только над 8-битными, но и над 16- или 32-битными регистрами, а операнды разных размеров должны быть корректно согласованы. Примером является ситуация, когда в программе имеется переменная, хранящаяся в одном байте (8 бит), но требуется использовать её в арифметических операциях вместе с другими числами, представленными в двух байтах (16 бит)(). Чтобы выполнить такие действия, нужно расширить значение хранящееся, в формате малой разрядности, до формата большей разрядности.

Существует два основных способа расширения чисел: знаковое расширение и беззнаковое расширение. При знаковом расширении важно сохранить правильное значение числа с учётом его знака. В архитектуре x86 отрицательные числа хранятся в формате дополнительного кода, а значит, если старший бит формата равен 1, то число можно воспринимать как отрицательное. При знаковом расширении процессор копирует этот старший бит во все старшие разряды нового регистра, чтобы сохранить числовое значение. Например, если в 8-битном регистре AL хранится число 11111111 (−1), то при расширении до AX получится 11111111 11111111 (−1 в 16-битном представлении). Если бы такое расширение выполнялось как беззнаковое, то мы получили бы значение 00000000 11111111, которое соответствует уже не −1, а +255. Таким образом, знаковое расширение гарантирует корректность результата при работе с числами, у которых учитывается знак.

Для выполнения знакового расширения в архитектуре процессоров 8086/80286/80386 предусмотрена специальная команда CBW (Convert Byte to Word). Она автоматически преобразует содержимое младшего 8-битного регистра AL в 16-битное значение AX, копируя бит знака во все старшие разряды. Эта операция является очень удобной и выполняется за один такт, что делает её предпочтительной при арифметических вычислениях. В более поздних процессорах (начиная с 80386) была добавлена команда MOVSX (Move with Sign-Extend), которая также позволяет скопировать 8-битное значение в 16-битный или 32-битный регистр с сохранением знака, также позволяет знаково скопировать 16-битное значение в 32-битный регистр.

Второй вариант расширения — это беззнаковое расширение, которое осуществляется путём заполнения новых разрядов нулями. Такой подход применяется тогда, когда число хранится в виде беззнакового целого, например, используется как код символа, индекс массива или маска. В этом случае старшие разряды заполняются нулями, и значение числа остаётся тем же самым. Для выполнения этого действия в архитектуре x86 начиная с процессора 80386 была предусмотрена команда MOVZX (Move with Zero-Extend), которая позволяет перенести содержимое младшего регистра в старший с добавлением нулей. Например, если в AL хранится число 11111111 (255), то после выполнения MOVZX AX, AL результат будет 00000000 11111111, то есть 255 в 16-битной форме.

Таким образом, выбор между MOVSX и MOVZX зависит от того, как именно интерпретируется число: как знаковое или беззнаковое. Если программист работает с математическими величинами, где возможны отрицательные значения, нужно использовать знаковое расширение. Если же обрабатываются коды символов, элементы таблиц или иные значения, которые по смыслу не могут быть отрицательными, следует использовать беззнаковое расширение.

Важно отметить, что в реальных программах такие преобразования встречаются постоянно. Например, при считывании символа из файла или клавиатуры результат хранится в одном байте, но при последующей обработке этот символ нужно использовать как часть выражений, где участвуют 16-битные регистры. Без корректного расширения вычисления привели бы к искажению результатов. Поэтому команды MOVSX и MOVZX можно рассматривать как базовые инструменты для работы с данными в TASM.

В заключение можно сказать, что расширение значений — это фундаментальная операция в ассемблере. Она отражает особенности архитектуры x86 и тесно связана с внутренним устройством процессора, в частности с представлением чисел и системой команд. В TASM программисту предоставляется возможность явно управлять типом расширения, что делает программы более гибкими и позволяет эффективно работать как с целыми числами со знаком, так и с беззнаковыми данными.

\section{Какими командами можно расширить двухбайтный код до 4-х байтного?}

При работе с ассемблером TASM и архитектурой процессоров семейства x86 программист часто сталкивается с задачей обработки данных разной разрядности. Если число хранится в двух байтах, то есть в формате слова, 16-битное представление, и требуется использовать его в более широком регистре, например, 32-битном регистре, необходимо выполнить расширение исходного 16-битного кода. Как и в случае с преобразованием из 8 бит в 16 бит, различают два вида расширения: знаковое и незнаковое.

Знаковое расширение используется для чисел со знаком, представленных в формате дополнительного кода. В таком случае старший бит 16-битного числа (бит знака) копируется во все старшие разряды 32-битного регистра. Благодаря этому сохраняется корректное числовое значение. Например, если в AX хранится число 11111111 11111111 (−1 в 16-битном представлении), то после знакового расширения до регистра EAX мы получим 11111111 11111111 11111111 11111111, то есть −1 в 32-битной форме. Если бы расширение было выполнено нулями, то результат оказался бы равен 65535, что уже полностью искажает исходное значение.

Для выполнения знакового расширения в процессоре 80386 и выше существует команда CWD (Convert Word to Doubleword). Она преобразует содержимое регистра AX в значение DX:AX, где DX получает копию знакового бита. Таким образом, пара регистров DX:AX образует 32-битное число. Однако в более поздних процессорах (80386 и выше) появилась более удобная команда CWDE (Convert Word to Doubleword in EAX), которая копирует 16-битное значение AX в 32-битный регистр EAX с автоматическим распространением знака. Благодаря этой команде работа с расширением стала проще и быстрее.

Для беззнаковых чисел применяется другой принцип: старшие разряды заполняются нулями. Этот метод известен как нулевое расширение. Он используется в тех случаях, когда число всегда положительно. Для выполнения такого преобразования в архитектуре 80386 и выше существует команда MOVZX (Move with Zero-Extend). Она позволяет скопировать значение AX в EAX с добавлением нулей в старшую часть. Например, если в AX хранится число 65535 (11111111 11111111), то после MOVZX результатом в EAX будет 00000000 00000000 11111111 11111111, то есть то же самое число в 32-битной форме.

Таким образом, выбор между знаковым и незнаковым расширением определяется контекстом использования кода числа. Если оно представляет математическую величину, где возможны отрицательные значения, применяется CWDE, CWD или MOVSX. Если же речь идёт о кодах символов, адресах или других беззнаковых данных, используется MOVZX.

Практическая значимость таких операций огромна. Во-первых, они позволяют обрабатывать данные разных форматов в рамках одной программы. Во-вторых, правильный выбор метода расширения напрямую влияет на корректность вычислений. Ошибка в выборе, например, использование нулевого расширения там, где нужно знаковое, может привести к неправильным результатам и логическим сбоям в программе.

Таким образом, команды CWD, CWDE, MOVZX и MOVSX являются ключевыми для работы с числами при расширении из 16 бит в 32 бита. Они отражают развитие архитектуры процессоров от 16-битных моделей (8086/80286) к полноценным 32-битным (80386 и выше). В TASM эти команды используются очень часто, так как большинство современных программных задач требует работы с данными переменной разрядности.


\section{Какие два значения являются для процессора программной <<координатой>> байта в пространстве физической памяти при сегментной адресации?}

Архитектура процессоров x86 основана на принципе сегментной организации памяти. В отличие от простого линейного подхода, где каждый байт имеет уникальный адрес, сегментная модель предполагает, что физический адрес байта вычисляется как комбинация двух величин: сегмента и смещения (offset). Именно эти два значения образуют то, что можно назвать «координатой» байта в пространстве физической памяти.

Первое значение — это сегментный адрес, который хранится в специальном сегментном регистре. Существует несколько сегментных регистров: CS (кодовый сегмент), DS (сегмент данных), SS (сегмент стека), ES, FS, GS (дополнительные сегменты). Каждый из них указывает на начало определённой области памяти, с которой работает программа. Например, CS определяет, где находится исполняемый код, DS — данные, а SS — стек.

Второе значение — это смещение (offset), то есть расстояние в байтах от начала сегмента до требуемого элемента памяти. Разрядность смещения зависит от объёма сегмента. Так как память в вычислительных системах с архитектурой фон Неймона определяется ячейками с бинарынм состоянием, то адресное пространство сегмента определяется через поле Галуа GF=2. Таким образом объём памяти в сегменте кратен степени двойки и представляет собой степень вида 2^N, соответственно смещение имеет формат N. Смещение задаётся  и вычисляется с помощью регистров общего назначения (BX, SI, DI, BP) или непосредственного значения. В ассемблере TASM смещения часто используются в косвенных и индексных адресациях.
Таким образом, для процессора каждый байт в памяти определяется двумя значениями: содержимым сегментного регистра и величиной смещения. Это и является его «координатой» в пространстве памяти.

Такой подход имеет ряд преимуществ. Во-первых, он позволяет программе работать с адресами, не выходящими за пределы 64 КБ, даже если физическая память значительно больше. Во-вторых, сегментная организация облегчает многозадачность и защиту памяти: можно выделять разные сегменты для разных частей программы и ограничивать доступ к ним.

С другой стороны, сегментная модель имеет и недостатки. Сложность вычисления физического адреса (сегмент * 16 + смещение) делает архитектуру менее прозрачной. Кроме того, при программировании на ассемблере нужно всегда помнить о том, какой сегментный регистр используется по умолчанию и как изменяется смещение.

Таким образом, ответ на вопрос заключается в том, что координатой байта при сегментной адресации являются сегментный адрес и смещение. Именно они вместе определяют физическое расположение данных в памяти, а сегментная модель памяти является одной из ключевых особенностей архитектуры x86.

\section{Что означает <<косвенная адресация>> операнда в памяти? Какие варианты косвенной адресации возможны?}

Косвенная адресация — это один из способов обращения к операндам в ассемблере, при котором команда не содержит непосредственно адрес данных, а указывает на регистр или комбинацию регистров, хранящих этот адрес. Иными словами, команда указывает не на сам операнд, а на «указатель» к нему. Такой метод используется для более гибкой работы с памятью, особенно когда данные расположены в массивах, таблицах или стеке.

Простейший пример косвенной адресации — это использование регистра BX. Если написать команду MOV AX, [BX], процессор возьмёт значение из памяти, по умолчанию в сегменте DS, по адресу, хранящемуся в регистре BX. Таким образом, BX играет роль указателя на элемент памяти.

Косвенная адресация может быть разной сложности. Помимо использования одного регистра, возможны комбинации базовых и индексных регистров. Базовыми регистрами считаются BX и BP, индексными — SI и DI. Дополнительно можно использовать смещения (константы).

Возможные варианты косвенной адресации в архитектуре x86:

[BX], [BP], [SI], [DI] — простая косвенная адресация.

[BX+SI], [BX+DI], [BP+SI], [BP+DI] — базово-индексная адресация.

[BX+смещение], [BP+смещение], [SI+смещение], [DI+смещение] — косвенная адресация с постоянным смещением.

[BX+SI+смещение] и аналогичные варианты — базово-индексная адресация со смещением.

Каждый из этих способов удобен в определённых ситуациях. Например, SI и DI часто используются при обработке строк и массивов, так как они могут автоматически увеличиваться или уменьшаться при выполнении команд REP MOVSB или REP SCASB. Использование BP удобно для работы со стеком и локальными переменными в процедурах.

Главное достоинство косвенной адресации заключается в её универсальности. Она позволяет строить гибкие структуры данных, работать с массивами переменной длины и обрабатывать динамические данные. В языке высокого уровня такие операции скрыты за синтаксисом массивов и указателей, но в ассемблере программист должен явно задавать, каким образом вычисляется адрес.

Таким образом, косвенная адресация операнда в памяти — это использование регистров и/или их комбинаций для вычисления адреса данных. Возможные варианты включают простую регистровую, базово-индексную, с постоянным смещением и с комбинацией регистра и смещения. Этот механизм является одним из важнейших в TASM и архитектуре x86.

\section {Перечислите имена 8-разрядных, 16-разрядных и 32-разрядных регистров процессора x386.}

Процессоры семейства x86 имеют развитую систему регистров, которые служат для хранения данных, адресов и управляющей информации. Каждый регистр имеет своё назначение и может использоваться как в общем виде, так и в узкоспециализированных командах. В TASM программисту необходимо хорошо знать структуру регистров, так как от этого зависит корректность написания программы.

8-разрядные регистры

Восьмиразрядные регистры представляют собой младшие и старшие части 16-битных регистров AX, BX, CX и DX.

AL (младшие 8 бит AX), AH (старшие 8 бит AX).

BL, BH — части BX.

CL, CH — части CX.

DL, DH — части DX.

Эти регистры применяются для хранения байтовых данных, работы с символами, побитовыми операциями и арифметикой над малыми величинами.

16-разрядные регистры

AX (аккумулятор) — основной регистр для арифметики и ввода-вывода.

BX (базовый регистр) — часто используется для адресации.

CX (счётчик) — применяется в циклах и сдвигах.

DX (регистр данных) — участвует в операциях умножения и деления.

SI (Source Index) — индекс источника для операций со строками.

DI (Destination Index) — индекс приёмника.

BP (Base Pointer) — базовый указатель, часто используется для доступа к локальным переменным.

SP (Stack Pointer) — указатель стека.

К ним добавляются сегментные регистры: CS, DS, SS, ES, FS, GS. Они определяют границы сегментов кода, данных и стека.

32-разрядные регистры (введены в 80386)

EAX, EBX, ECX, EDX — расширенные версии 16-битных регистров.

ESI, EDI, EBP, ESP — расширенные версии индексных и указательных регистров.

EIP — указатель команд, хранит адрес следующей инструкции.

EFLAGS — регистр флагов, хранящий результаты операций и управляющие биты.

Знание структуры регистров крайне важно, потому что многие команды TASM жестко привязаны к конкретным регистрами. Например, команда MUL при умножении всегда использует AX или EAX. Команды циклов ориентируются на CX или ECX. Сегментные регистры также используются по умолчанию (например, при косвенной адресации через BX или SI по умолчанию применяется DS).

Таким образом, регистры процессора x386 делятся на три группы: 8-разрядные (AH, AL, BH и др.), 16-разрядные (AX, BX, CX, DX, SI, DI и др.) и 32-разрядные (EAX, EBX, ECX и др.). Каждая из этих групп имеет своё назначение, и эффективное владение ими является основой программирования на TASM.

\endinput
                                     % Первая глава
\chapter{Арифметические и логические операции}
\begin{itemize}
\item Чем отличается команда SBB от SUB?
\item Как выполняется команда CMP?
\item Как выполняется процессором команда NEG?
\item Каково различие между длинным и коротким прямым переходом?
\item Чем отличаются логические команды от арифметических команд?
\item Чем отличаются команды логического умножения AND и TEST?
\item Какой логической командой можно обнулить отдельные биты регистра или ячейки памяти?
\item Какой логической командой можно установить отдельные биты регистра или ячейки памяти в 1?
\item Какой логической командой можно инвертировать отдельные биты регистра или ячейки памяти?
\item В каких случаях с помощью команд сдвига можно выполнять умножение и деление?
\item Каковы особенности выполнения операций обычного, арифметического и циклического сдвига?
\end{itemize}

\section{Чем отличается команда SBB от SUB?}

В архитектуре процессоров семейства x86 и языке ассемблера TASM обе команды — SBB и SUB — предназначены для выполнения операции вычитания. Однако их различие заключается в том, каким образом они учитывают флаг переноса (CF, Carry Flag), который хранится в регистре флагов процессора и отражает результат предыдущей арифметической операции.
Команда SUB (Subtract) выполняет простое вычитание второго операнда из первого без учёта каких-либо предыдущих действий. Например, при выполнении SUB AX, BX из содержимого регистра AX вычитается значение BX, а результат записывается обратно в AX. После выполнения команда изменяет флаги процессора, такие как флаг нуля (ZF), знака (SF), переполнения (OF), переноса (CF) и других. Однако SUB не использует значение флага CF при вычислении — она его только обновляет в зависимости от результата вычитания.
Команда SBB (Subtract with Borrow), напротив, предназначена для вычитания с учётом возможного займа из предыдущей операции. При её выполнении из первого операнда вычитается не только значение второго операнда, но и значение флага CF. То есть фактически выполняется операция:
A ← A − B − CF.
Если флаг CF равен 1 (что означает, что в предыдущем вычитании был «заём»), то из результата дополнительно вычитается единица.
Это различие имеет принципиальное значение при выполнении многобайтовых и многоразрядных вычитаний. Например, если нужно вычесть одно 32-битное число из другого в 16-битной архитектуре, операция делится на два этапа: сначала вычитаются младшие 16 бит с помощью SUB, затем старшие 16 бит — с помощью SBB, которая учитывает возможный «заём» из младшей части. Таким образом обеспечивается корректное арифметическое поведение при работе с числами, превышающими разрядность регистра.
Команда SBB особенно важна при реализации операций с длинными целыми числами, алгоритмов шифрования, арифметики с плавающей точкой в программном виде и других ситуациях, где требуется точное управление переносами между частями числа.
Таким образом, основное различие между командами SUB и SBB заключается в учёте флага CF. SUB выполняет простое вычитание без учёта предыдущих операций, тогда как SBB используется для последовательных вычитаний, когда необходимо учитывать перенос (или заём) из предыдущего разряда. Это различие делает SBB ключевой частью всех алгоритмов работы с числами большей разрядности, чем может вместить регистр процессора.

\section{Как выполняется команда CMP}

Команда CMP (Compare) — одна из базовых инструкций языка ассемблера TASM, предназначенная для сравнения двух операндов. Однако в действительности команда не выполняет непосредственного сравнения в логическом смысле, а реализуется как операция вычитания без сохранения результата. Это означает, что CMP вычитает второй операнд из первого, но не изменяет ни один из них. Её основная цель — обновить флаги процессора в зависимости от результата этой операции, чтобы на их основе можно было выполнить условные переходы.
Фактически выполнение CMP можно представить как:
A − B,но результат этого вычитания не сохраняется в A. Вместо этого процессор обновляет флаги:
ZF (Zero Flag) — устанавливается, если результат равен нулю (то есть операнды равны).
SF (Sign Flag) — отражает знак результата (если результат отрицательный, SF = 1).
CF (Carry Flag) — показывает, был ли «заём» при вычитании (если A < B, CF = 1).
OF (Overflow Flag) — устанавливается при арифметическом переполнении.
После выполнения CMP программа обычно использует результаты сравнения в командах условного перехода, таких как JE (Jump if Equal), JNE (Jump if Not Equal), JL (Jump if Less), JG (Jump if Greater) и других. Таким образом, CMP и последующий условный переход образуют вместе аналог оператора if в языках высокого уровня.
Команда CMP не изменяет содержимого операндов, что делает её безопасной для использования в циклах, проверках и логических ветвлениях. Она может применяться как для чисел со знаком, так и без знака. Тип сравнения определяется тем, какие именно команды перехода используются после неё: например, JA/JB (above/below) — для беззнаковых чисел, JG/JL — для знаковых.
Таким образом, CMP выполняет вычитание для определения отношения между двумя операндами и изменяет флаги процессора, не изменяя при этом сами операнды. Это делает её центральным инструментом для организации ветвлений, циклов и проверок в программах на ассемблере.

\section{Как выполняется процессором команда NEG?}

Команда NEG (Negate) предназначена для получения отрицательного значения операнда, то есть вычисления его арифметического отрицания. Она реализует операцию, эквивалентную вычитанию операнда из нуля:
Операнд ← 0 − Операнд.
При выполнении этой инструкции процессор инвертирует все биты числа и добавляет единицу (что соответствует операции получения дополнительного кода). Таким образом, если исходное значение было положительным, результат становится отрицательным, и наоборот. Например, если значение было 5, после NEG оно становится −5; если −3 — становится 3.
В процессе выполнения NEG изменяются флаги процессора:
CF (Carry Flag) устанавливается, если операнд был отличен от нуля (так как происходит заём при вычитании).
ZF (Zero Flag) устанавливается, если результат равен нулю (например, при NEG 0).
SF (Sign Flag) отражает знак результата.
OF (Overflow Flag) устанавливается при переполнении (например, при попытке изменить знак числа с минимальным отрицательным значением, которое не имеет положительного аналога).
Команда NEG часто используется для изменения направления арифметических вычислений, при реализации операций вычитания через сложение, а также в операциях, связанных с обработкой знаковых данных. Например, если нужно заменить SUB AX, BX на эквивалентную последовательность, можно выполнить NEG BX и затем ADD AX, BX.
Таким образом, команда NEG выполняет инвертирование знака числа и реализуется как операция вычитания из нуля. Она изменяет флаги процессора и является основным средством для получения противоположных по знаку значений в ассемблерных программах.

\section{Каково различие между длинным и коротким прямым переходом?}

В ассемблере TASM, как и в архитектуре x86, команды переходов позволяют изменять последовательность выполнения программы. Существует несколько разновидностей переходов, в том числе короткие (short) и длинные (near/far). Их различие связано с тем, как вычисляется адрес перехода и какое расстояние между командами допускается.
Короткий переход (short jump) используется, когда целевая инструкция находится в пределах −128…+127 байт от текущего адреса. Это наиболее экономичный тип перехода, поскольку команда занимает всего 2 байта: один — на код операции, второй — на смещение относительно текущего адреса. Короткие переходы применяются в пределах небольших участков программы, например, внутри циклов или условных блоков.
Длинный (near) переход позволяет перейти к любой команде в пределах текущего сегмента. В этом случае смещение указывается полностью (16 бит для реального режима, 32 бита для защищённого). Команда занимает больше места — 3 байта (в 16-битном режиме) или 5 байт (в 32-битном).
Существует также дальний (far) переход, который используется для перехода в другой сегмент памяти. Он содержит не только смещение, но и значение сегмента, поэтому команда занимает 5 байт в 16-битном режиме (2 байта на смещение, 2 — на сегмент, плюс код операции).
Таким образом, различие между коротким и длинным переходом заключается в размере адресного смещения и дальности, на которую возможен переход. Короткий — наиболее экономичный, но ограничен по диапазону. Длинный — универсальный в пределах сегмента, а дальний — позволяет переходить между сегментами программы.

\section{Чем отличаются логические команды от арифметических команд?}

Логические и арифметические команды в языке ассемблера TASM выполняют разные типы операций над двоичными данными, хотя внешне они могут выглядеть схожими. Основное различие между ними заключается в том, что именно они делают с битами данных и какие флаги процессора изменяют.
Арифметические команды (например, ADD, SUB, INC, DEC, MUL, DIV, NEG, ADC, SBB) выполняют математические операции — сложение, вычитание, умножение, деление, инкремент, декремент и т.д. Эти команды интерпретируют данные как числа (со знаком или без) и учитывают переносы и переполнения. При их выполнении изменяются флаги, связанные с результатом вычислений:
CF (Carry Flag) — устанавливается, если при сложении или вычитании произошёл перенос или заём;
OF (Overflow Flag) — сигнализирует о переполнении при операциях со знаковыми числами;
ZF (Zero Flag) — показывает, равен ли результат нулю;
SF (Sign Flag) — определяет знак результата.
Логические команды (AND, OR, XOR, NOT, TEST) рассматривают данные не как числа, а как набор отдельных битов. Они выполняют операции булевой логики: конъюнкцию, дизъюнкцию, исключающее ИЛИ и отрицание. Эти операции не связаны с понятием переноса, поэтому CF и OF всегда сбрасываются в 0, а остальные флаги (ZF, SF, PF) устанавливаются в зависимости от результата.
Таким образом, логические команды используются для побитовой обработки данных — включения или обнуления отдельных битов, проверки флагов, маскирования и подготовки данных для последующих арифметических или управляющих операций. Арифметические же команды — для изменения численных значений и выполнения вычислений.
Проще говоря, арифметические инструкции оперируют значениями, а логические — их структурой. Именно благодаря такому разделению ассемблер позволяет эффективно управлять не только данными, но и самими битами, из которых они состоят.

\section{Чем отличаются команды логического умножения AND и TEST?}

Команды AND и TEST в TASM действительно похожи, поскольку обе выполняют операцию побитового логического И (AND). Различие между ними заключается не в самой логике вычислений, а в том, что происходит с результатом.
Команда AND выполняет логическое умножение (побитовое И) двух операндов и сохраняет результат в первом операнде. Например:
AND AX, 0FFh — в результате в AX останется значение, у которого обнулены все старшие биты.
Эта операция используется для очистки, маскирования или выделения определённых битов. После выполнения команда изменяет флаги:
ZF — устанавливается, если результат равен нулю;
SF — отражает знак результата;
PF — показывает чётность количества единиц в результате;
CF и OF всегда сбрасываются в 0.
Команда TEST, напротив, выполняет такое же побитовое И, но результат нигде не сохраняет. Она предназначена исключительно для проверки определённых битов без изменения данных. Например:
TEST AL, 1 — проверяет, установлен ли младший бит регистра AL. Если результат нулевой, значит бит равен 0; иначе — 1.
Таким образом, команда TEST — это безопасный способ «заглянуть» в данные без их изменения, а AND — инструмент для реального изменения содержимого регистра или ячейки памяти.

\section{ Какой логической командой можно обнулить отдельные биты регистра или ячейки памяти?}

Для обнуления отдельных битов регистра или ячейки памяти используется команда AND. Логическое умножение (AND) выполняется по битам, и в результате каждый бит становится равным 1 только если оба соответствующих бита операндов равны 1. Если один из них равен 0, то результатом будет 0.
Поэтому если нужно обнулить определённые биты, применяется маска, в которой те биты, что должны остаться, устанавливаются в 1, а те, которые нужно сбросить, устанавливаются в 0.
Это свойство делает команду AND одним из ключевых инструментов при низкоуровневой обработке данных, особенно в драйверах, микроконтроллерах, обработке флагов и кодировании состояний устройств.

\section{Какой логической командой можно установить отдельные биты регистра или ячейки памяти в 1?}

Для установки отдельных битов в единицу используется команда OR. Она выполняет побитовую дизъюнкцию (логическое ИЛИ), при которой бит результата равен 1, если хотя бы один из соответствующих битов операндов равен 1.
Если нужно установить конкретные биты, используется маска, в которой биты, подлежащие установке, равны 1, а остальные — 0.
Команда OR часто используется для задания флагов, включения режимов работы устройств, формирования управляющих байтов и настройки параметров, когда требуется установить конкретные разряды, не влияя на другие.

\section{Какой логической командой можно инвертировать отдельные биты регистра или ячейки памяти?}

Для инверсии битов используется команда NOT. Она выполняет побитовое отрицание, т.е. заменяет все нули на единицы и все единицы на нули. Если исходное значение было 10101010b, то после выполнения команда даст 01010101b
Инструкция NOT не изменяет никаких флагов процессора, что отличает её от большинства других логических операций. Она часто применяется при операциях с масками, при побитовом кодировании и в логике, где требуется обращение всех битов, например, при формировании обратных кодов или вычислении отрицаний булевых выражений.

\section{В каких случаях с помощью команд сдвига можно выполнять умножение и деление?}

Команды сдвига (SHL, SHR, SAL, SAR) позволяют не только перемещать биты влево или вправо, но и выполнять эффективные операции умножения и деления на степени числа 2.
Если выполнить сдвиг влево (SHL или SAL), каждый бит числа перемещается на один разряд влево, а младший заполняется нулём. Это эквивалентно умножению числа на 2. Если сдвинуть на 2 позиции — умножаем на 4, на 3 — на 8 и так далее. Например:
SHL AX, 1 — умножает содержимое регистра AX на 2.
Аналогично, сдвиг вправо (SHR) делит беззнаковое число на 2, а SAR — делит знаковое число, сохраняя его знак.
Эти операции выполняются значительно быстрее, чем обычное умножение или деление, и используются в оптимизированных программах, где важна производительность — например, при графических вычислениях, обработке массивов или в системном программировании.

\section{Каковы особенности выполнения операций обычного, арифметического и циклического сдвига?}

В архитектуре x86 существует три основных типа сдвига: логический, арифметический и циклический. Все они перемещают биты, но делают это по-разному, в зависимости от того, как обращаются со знаковыми и крайними битами.
Логический сдвиг (SHL, SHR) — сдвигает биты влево или вправо, а освободившиеся позиции заполняет нулями. Он подходит для работы с беззнаковыми числами.
Арифметический сдвиг (SAL, SAR) — применяется к числам со знаком. При сдвиге вправо сохраняется старший (знаковый) бит, что позволяет сохранять знак числа. Например, при делении отрицательных чисел результат остаётся отрицательным.
Циклический сдвиг (ROL, ROR, RCL, RCR) — переносит биты, «выпавшие» с одного края, на противоположный. Это используется при криптографических преобразованиях, шифровании, побитовых операциях с ключами и контрольных суммах.
Циклические сдвиги бывают с участием флага переноса (RCL, RCR) и без него (ROL, ROR), что позволяет гибко управлять движением битов.
Таким образом, различие между типами сдвига заключается в том, как они обращаются с освободившимися битами и сохраняют ли знак числа. Эти инструкции позволяют реализовывать целые классы математических, логических и криптографических операций без использования сложных арифметических команд.

\endinput
                                     % Вторая глава
\chapter{Разветвления}
\begin{itemize}
\item Чем похожи команды вычитания SUB и сравнения CMP?
\item Почему используются разные команды условного перехода после сравнения знаковых и беззнаковых чисел?
\item В командах прямых внутрисегментных переходов расстояние перехода отсчитывается от \dots ?
\item Каково различие между длинным и коротким прямым переходом?
\end{itemize}

\section{ем похожи команды вычитания SUB и сравнения CMP?}

Команды SUB (Subtract) и CMP (Compare) в языке ассемблера TASM имеют много общего, поскольку обе основаны на одной и той же арифметической операции — вычитании. В обоих случаях процессор вычисляет разность между двумя операндами, однако различие между ними заключается в том, куда помещается результат и для чего используется команда.

Команда SUB выполняет арифметическую операцию вычитания:
Операнд1 ← Операнд1 − Операнд2.
Результат сохраняется в первом операнде, а также обновляются все основные флаги процессора: флаг переноса (CF), знака (SF), нуля (ZF), переполнения (OF), паритета (PF) и вспомогательный флаг (AF). Эта команда изменяет данные, поскольку её цель — выполнить реальное вычисление.

Команда CMP, напротив, предназначена исключительно для сравнения двух значений. Она выполняет ту же арифметическую операцию вычитания, но не сохраняет результат — он используется только для обновления флагов процессора. По сути, CMP можно рассматривать как “SUB без записи результата”.

После выполнения CMP процессор анализирует флаги, чтобы определить, какое из чисел больше, меньше или равно другому. Например:

если ZF = 1, значит, операнды равны;

если CF = 1, значит, первое число меньше второго (для беззнаковых чисел);

если SF ≠ OF, значит, первое число меньше второго (для знаковых чисел).

Таким образом, CMP используется вместе с командами условных переходов (JE, JNE, JA, JB, JG, JL и др.), которые интерпретируют состояние флагов и определяют, выполнять ли переход.

Обе команды тесно связаны: CMP — это «чистое» сравнение без изменения данных, а SUB — реальное вычитание. Поэтому на уровне микрокода они работают одинаково, но с разной целью. Такое разделение позволяет ассемблеру быть более гибким: SUB используют для вычислений, CMP — для принятия решений и ветвлений программы.

\section{Почему используются разные команды условного перехода после сравнения знаковых и беззнаковых чисел?}

В архитектуре процессора x86 различаются два типа чисел — знаковые и беззнаковые, и это различие напрямую влияет на интерпретацию результатов сравнения. Дело в том, что знаковое число хранится в дополнительном коде, где старший бит (бит 7 в байте, бит 15 в слове) служит признаком знака: 0 — положительное, 1 — отрицательное.

Из-за этого два одинаковых двоичных значения могут иметь совершенно разный смысл. Например, байт 10000000b — это 128 в беззнаковом представлении и −128 в знаковом. Поэтому процессор должен по-разному трактовать результаты вычитания в зависимости от того, как разработчик рассматривает данные.

После выполнения команды CMP флаги процессора (CF, ZF, SF, OF) изменяются одинаково, но значение этих флагов нужно интерпретировать по-разному для знаковых и беззнаковых чисел:

Для беззнаковых чисел основным является флаг переноса (CF), который показывает, был ли «заём». Если CF = 1, значит, первое число меньше второго.
Поэтому для беззнаковых сравнений используются команды:

JA (Jump Above) — переход, если больше (CF = 0 и ZF = 0);

JB (Jump Below) — переход, если меньше (CF = 1);

JAE, JBE — варианты с равенством.

Для знаковых чисел важен не CF, а комбинация SF (Sign Flag) и OF (Overflow Flag), которые позволяют определить результат в дополнительном коде.
Используются команды:

JG (Jump Greater) — переход, если больше (ZF = 0 и SF = OF);

JL (Jump Less) — переход, если меньше (SF ≠ OF);

JGE, JLE — варианты с равенством.

Таким образом, различие в командах перехода связано не с самой операцией CMP, а с тем, как интерпретируются её результаты. Для беззнаковых чисел используется CF, для знаковых — SF и OF. Это обеспечивает корректную работу программы независимо от того, является ли число в памяти положительным или отрицательным.

\section{В командах прямых внутрисегментных переходов расстояние перехода отсчитывается от \dots ?}

В командах прямых внутрисегментных переходов (direct near jumps) расстояние перехода отсчитывается от адреса следующей за командой инструкции, то есть от точки, куда указывает счётчик команд (IP/EIP) после выборки текущей команды.

Когда процессор выполняет инструкцию перехода, он уже считает адрес следующей команды, увеличивая IP (в 16-битном режиме) или EIP (в 32-битном). Поэтому значение смещения, закодированное в переходе, прибавляется не к адресу самой команды перехода, а к адресу следующей команды.
Здесь команда JMP SHORT 0107h имеет смещение +5. Оно отсчитывается от 0102h — адреса следующей инструкции после JMP.

Такое поведение объясняется принципом относительной адресации (relative addressing), при котором переход задаётся не абсолютным адресом, а смещением относительно текущего положения. Это упрощает компоновку и загрузку программы, поскольку переход не зависит от того, где именно в памяти размещён код.

Таким образом, при прямом внутрисегментном переходе процессор вычисляет новый адрес как:
IP_new = IP_current + 2 + displacement
(для короткого перехода, где 2 — длина команды).

Такое решение повышает гибкость и переносимость машинного кода, позволяя перемещать блоки программы без пересчёта всех переходов.

\section{Каково различие между длинным и коротким прямым переходом?}

Команды прямого перехода (JMP, JE, JNE, JG, JL и др.) могут быть короткими (short) или длинными (near). Разница между ними определяется диапазоном адресов, на который можно перейти, и количеством байтов, занимаемых командой в памяти.

Короткий (short) переход используется, когда целевая команда находится на расстоянии от −128 до +127 байт от текущей. В этом случае в машинном коде сохраняется только 1-байтовое смещение, а сама команда занимает всего 2 байта (код операции + смещение).
Преимущество короткого перехода — компактность и высокая скорость выполнения. Недостаток — ограниченная дальность.

Длинный (near) переход позволяет перемещаться на любое расстояние в пределах текущего сегмента (до 64 Кбайт в реальном режиме). Здесь смещение занимает 16 или 32 бита, а команда — 3–5 байт. Такой переход используется, если цель находится дальше, чем допускает короткий формат.

Существуют также дальние (far) переходы, которые изменяют не только смещение (IP/EIP), но и сегментный регистр CS. Они используются для перехода между сегментами кода и занимают больше байтов (5 в 16-битном режиме).

Таким образом, различие между коротким и длинным переходом состоит в:

Размере смещения (1 байт против 2/4 байт);

Диапазоне перехода (±128 байт против всего сегмента);

Объёме занимаемой памяти;

Скорости выполнения — короткие переходы немного быстрее.

Компилятор и ассемблер TASM обычно выбирают короткий переход автоматически, если адрес цели попадает в допустимый диапазон. Если нет — применяется длинный. Такое различие позволяет одновременно сохранять компактность программы и обеспечивать возможность перехода на любой адрес внутри сегмента.

\endinput
                                     % Третья глава
\chapter{Организация циклов. Массивы данных}
\begin{itemize}
\item Что называют <<массивом данных>>? Что отличает каждый <<элемент>> массива в памяти?
\item Какой способ внутрисегментной адресации можно использовать, чтобы организовать цикл действий над массивом данных в памяти?
\item Какие регистры можно использовать для задания адреса в этих способах адресации?
\item Какой командой можно занести в регистр:
    \begin{itemize}
    \item символический внутрисегментный адрес;
    \item числовое значение адреса?
    \end{itemize}
\item Как выполняется команда LOOP? Какой регистр используется ею в качестве вычитающего счётчика циклов?
\item Можно ли организовать цикл без использования команды LOOP?
\end{itemize}

\section{Что называют <<массивом данных>>? Что отличает каждый <<элемент>> массива в памяти?}

Массив данных — это упорядоченная совокупность однотипных элементов, размещённых в памяти подряд, без промежутков, и имеющих общее имя (метку). Каждый элемент массива имеет одинаковый размер (1, 2, 4 или больше байт), а различие между ними определяется их смещением относительно начала массива.

В языке ассемблера TASM массивы определяются с помощью директив резервирования памяти, таких как DB (define byte), DW (define word), DD (define double word) и т. д.
Особенности элементов массива заключаются в том, что каждый из них:

-Имеет уникальный адрес в памяти, который вычисляется как:Адрес_элемента = Базовый_адрес_массива + Смещение.
-Обладает фиксированным размером, одинаковым для всех элементов массива.
-Может быть доступен по индексу, который отражает его позицию в массиве.

Таким образом, отличие каждого элемента массива заключается в его смещении от начала массива, то есть в разнице адресов. При последовательной обработке массива процессор увеличивает указатель (индекс) на размер одного элемента и таким образом переходит к следующему элементу.

Массивы — основа для организации циклов, таблиц, строк и структур данных в ассемблерных программах. Они обеспечивают удобный способ хранения и последовательной обработки однотипных данных, например, при работе с текстом, числами, кодами или управляющими таблицами.

\section{Какой способ внутрисегментной адресации можно использовать, чтобы организовать цикл действий над массивом данных в памяти?}

Для организации циклов обработки массивов в памяти используется косвенная адресация (indirect addressing). Её суть заключается в том, что в регистре хранится не само значение данных, а адрес элемента массива, по которому нужно произвести операцию.
Например, если нужно последовательно обработать массив байтов, можно использовать команду:

MOV AL, [BX]

Здесь процессор интерпретирует содержимое регистра BX как адрес элемента массива внутри сегмента данных (DS).

Чтобы перейти к следующему элементу массива, достаточно изменить значение BX — увеличить его на размер элемента. Таким образом, при повторении цикла значение BX автоматически «указует» на следующий элемент массива.
Помимо простой косвенной адресации, часто применяется индексная адресация — когда адрес вычисляется как сумма базового адреса и индекса элемента:

MOV AL, [BX+SI]
или
MOV AX, [BP+DI]

Этот способ особенно удобен при работе с двухмерными массивами, таблицами структур или при параллельной обработке данных.

Таким образом, основной способ внутрисегментной адресации для организации циклов над массивами — косвенный и индексный, поскольку они позволяют изменять адрес каждого элемента во время выполнения программы и эффективно обходить массив в цикле.

\section{Какие регистры можно использовать для задания адреса в этих способах адресации?}

В архитектуре процессора x86 существует несколько адресных регистров, специально предназначенных для работы с памятью. Они могут использоваться по отдельности или в сочетании для различных способов адресации.

Для внутрисегментной адресации применяются следующие регистры:

BX (Base Register) — базовый регистр данных. Чаще всего используется для указания начала массива.

BP (Base Pointer) — базовый указатель стека или структур данных, часто используется вместе с сегментом SS.

SI (Source Index) — индексный регистр источника, применяется для чтения данных из массива.

DI (Destination Index) — индексный регистр при записи или обработке данных.

Для операций над массивами в сегменте данных (DS) обычно используются BX, SI, DI. При этом возможны различные комбинации:

MOV AL, [BX]         ; косвенная адресация
MOV AL, [BX+SI]      ; базово-индексная адресация
MOV AL, [SI+5]       ; индексная адресация со смещением

Все эти регистры обеспечивают гибкую адресацию, позволяя задавать не только текущее положение в массиве, но и шаг перехода между элементами.

Таким образом, для задания адреса при работе с массивами применяются регистры BX, BP, SI, DI, которые могут использоваться как отдельно, так и в комбинациях, образуя различные режимы внутрисегментной адресации.

\section{Какой командой можно занести в регистр}

Для занесения символического внутрисегментного адреса (то есть адреса, заданного через метку в программе) используется команда LEA (Load Effective Address). Она загружает в регистр вычисленный адрес (смещение) операнда, не обращаясь к самой памяти.

Например:

LEA BX, ARRAY

В результате выполнения этой команды в регистр BX будет помещено смещение начала массива ARRAY в сегменте данных, но не его содержимое. Команда LEA используется, когда нужно работать с адресами, например, для итерации по массиву или передачи указателя в подпрограмму.

Для занесения числового значения адреса используется команда MOV. Она позволяет напрямую записать константу в регистр:

MOV BX, 0020h
В этом случае в BX будет находиться просто число 0020h, которое может быть как адресом, так и обычным значением.

Таким образом:

Для символического адреса — LEA (получает адрес по имени метки);
Для числового значения — MOV (записывает константу напрямую).

\section{Как выполняется команда LOOP? Какой регистр используется ею в качестве вычитающего счётчика циклов?}

Команда LOOP — это специальная инструкция для организации циклов, которая автоматически уменьшает счётчик итераций и выполняет переход, пока счётчик не станет равным нулю.

Её синтаксис:

LOOP метка

При выполнении команда уменьшает содержимое регистра CX (в 16-битном режиме) или ECX (в 32-битном). После этого процессор проверяет значение регистра:

если оно не равно нулю, выполняется переход по указанной метке;

если равно нулю — выполнение продолжается со следующей команды.

Пример:

MOV CX, 5
M1:  ADD AL, [BX]
     INC BX
     LOOP M1

Здесь команда LOOP создаёт цикл, который выполняется 5 раз. Каждый раз CX уменьшается на 1.

Команда LOOP удобна своей простотой — она объединяет в себе уменьшение счётчика и проверку условия. Однако, в отличие от условных переходов, она не анализирует флаги, а работает только с CX/ECX.

Таким образом, регистр CX (или ECX) используется как счётчик циклов, а команда LOOP автоматически организует повторение блока команд заданное количество раз.

\section{Можно ли организовать цикл без использования команды LOOP?}

Да, цикл в языке ассемблера можно организовать без использования команды LOOP, применяя обычные команды сравнения (CMP) и условные переходы (Jxx). Такой способ более гибкий, поскольку позволяет задавать любые условия выхода из цикла, а не только зависимость от CX.

Пример цикла без LOOP:

MOV CX, 5
M1:  ADD AL, [BX]
     INC BX
     DEC CX
     JNZ M1

Здесь команда DEC CX уменьшает счётчик вручную, а JNZ (Jump if Not Zero) выполняет переход, пока CX не станет равным нулю. По сути, это аналог команды LOOP, но реализованный через общие инструкции.

Такой подход имеет несколько преимуществ:

можно использовать любой регистр в качестве счётчика;

можно задать сложные условия выхода (например, по флагу ZF, SF и др.);

цикл можно сделать вложенным или зависимым от данных, а не от фиксированного числа итераций.

Таким образом, команда LOOP — удобный инструмент для простых циклов, но далеко не единственный. Практически любой цикл в ассемблере можно построить с помощью комбинации CMP + условного перехода, что делает программную логику более гибкой и универсальной.

\endinput

\chapter{Процедуры}
\begin{itemize}
\item С какой целью создают подпрограммы (процедуры) при разработке программной логики?
\item Как задавать входные и где размещать выходные параметры для процедуры в ассемблерной программе?
\item Какая команда должна быть последней исполняемой в процедуре?
\item Что такое макрокоманда транслятора?
\item Для чего создают макрокоманды? Чем макрокоманда отличается от процедуры?
\end{itemize}
\endinput
\chapter{Форматы команд. Трансляция символической команды в код}
\begin{itemize}
\item Какова минимальная длина команды в байтах?
\item От чего зависит длина команды с операндами?
\item Что такое <<вторичный>> код операции? В формате каких команд он присутствует?
\item Приведите пример команды с максимальной длиной.
\end{itemize}
\endinput
\chapter{Использование системного сервиса. Ввод данных с клавиатуры. Вывод на экран}
\begin{itemize}
\item Каковы источники и типы прерываний? Каков механизм выполнения прерываний?
\item Опишите алгоритмы преобразования символьных данных в числовые и обратно.
\item Какие существуют функции системного сервиса прерывания Int 21h для организации ввода с клавиатуры и вывода на экран?
\item Как организовать программный вызов системного сервиса?
\end{itemize}
\endinput
\chapter{Использование системного сервиса. Работа с файлами}
\begin{itemize}
\item Какие функции системного сервиса прерывания 21h необходимы для работы с
файлами данных?
\item Что такое дескриптор файла?
\item Расшифруйте CON, AUX, PRN и DTA.
\item Каково назначение указателя файла?
\end{itemize}

section{Какие функции системного сервиса прерывания 21h необходимы для работы с файлами данных?}

Работа с файлами в операционной системе MS-DOS осуществляется через вызовы системного прерывания INT 21h, которое предоставляет целый набор функций для открытия, создания, чтения, записи и закрытия файлов. Эти сервисы обеспечивают взаимодействие прикладных программ с файловой системой FAT, скрывая аппаратные детали.

Основные функции работы с файлами:

Создание файла (функция 3Ch)
В регистре AH = 3Ch, в CX задаются атрибуты файла (обычно 0 — обычный файл), а в DS:DX передаётся адрес ASCIIZ-строки с именем файла.
В результате в AX возвращается дескриптор файла — уникальный номер, по которому система идентифицирует открытый файл.

Открытие существующего файла (функция 3Dh)
В AH = 3Dh, в AL — режим доступа (0 = только чтение, 1 = только запись, 2 = чтение/запись), а в DS:DX — адрес имени файла.
Возвращает дескриптор файла в AX.

Закрытие файла (функция 3Eh)
В AH = 3Eh, в BX — дескриптор файла.
Закрывает файл, освобождая ресурсы, связанные с ним.

Чтение из файла (функция 3Fh)
В AH = 3Fh, BX — дескриптор файла, CX — количество байт для чтения, DS:DX — адрес буфера.
После вызова INT 21h в AX возвращается количество реально считанных байт.

Запись в файл (функция 40h)
В AH = 40h, BX — дескриптор файла, CX — количество записываемых байт, DS:DX — адрес данных.
После выполнения возвращается количество записанных байт в AX.

Удаление файла (функция 41h)
В AH = 41h, в DS:DX — адрес имени удаляемого файла.
После вызова файл удаляется из каталога.

Позиционирование (перемещение указателя) в файле (функция 42h)
В AH = 42h, BX — дескриптор файла, AL — способ смещения (0 = от начала, 1 = от текущего положения, 2 = от конца), CX:DX — значение смещения.
После выполнения новый указатель позиции возвращается в DX:AX.

Эти функции позволяют реализовать полный цикл работы с файлами: создание, запись, чтение и закрытие. Все операции с файлами требуют обращения через дескриптор — системный идентификатор, по которому DOS управляет файлами

section{Что такое дескриптор файла?}

Дескриптор файла — это числовой идентификатор, который операционная система MS-DOS присваивает каждому открытому файлу. Этот идентификатор используется программой при выполнении операций чтения, записи или закрытия, вместо непосредственного указания имени файла.

Когда программа вызывает системный сервис INT 21h для открытия или создания файла (функции 3Dh и 3Ch), DOS выделяет для него запись в таблице открытых файлов. В этой записи хранятся:

адрес начала файла на диске;

текущее положение указателя чтения/записи;

атрибуты доступа;

информация о состоянии файла (например, конец файла, ошибки).

Возвращаемое значение — дескриптор (File Handle) — это индекс этой записи. Обычно DOS возвращает маленькое целое число (0, 1, 2 и далее).

Примеры стандартных дескрипторов:

0 — стандартный ввод (клавиатура, CON);

1 — стандартный вывод (экран, CON);

2 — стандартный вывод ошибок (CON).

Таким образом, дескриптор — это средство связи между прикладной программой и внутренними структурами DOS, обеспечивающее эффективную и абстрактную работу с файлами без обращения к физическим адресам носителя.

section{Расшифруйте CON, AUX, PRN и DTA.}

Эти обозначения являются специальными именами устройств DOS, которые система обрабатывает так же, как обычные файлы. Они позволяют обращаться к устройствам ввода-вывода единым способом — через файловые функции (INT 21h).

1. CON (Console)

Обозначает консоль — стандартные устройства ввода и вывода (клавиатура и экран).

Чтение из CON — это ввод с клавиатуры.

Запись в CON — это вывод текста на экран.
Пример: COPY file.txt CON выведет содержимое файла на экран.

2. AUX (Auxiliary)

Представляет дополнительное устройство — чаще всего последовательный порт (COM1).
Используется для связи с модемами и другими последовательными устройствами.

3. PRN (Printer)

Обозначает принтер, обычно подключённый к LPT1.
Запись в PRN приводит к выводу данных на печать.

4. DTA (Disk Transfer Area)

Это область памяти, используемая DOS при операциях с каталогами и файлами.
При вызове функций поиска (например, 4Eh — Find First, 4Fh — Find Next) DOS записывает найденные данные (имя, атрибуты, размер, дату) в DTA.
Адрес DTA можно задать вручную функцией 1Ah прерывания 21h.

Таким образом, CON, AUX, PRN — это логические устройства ввода-вывода, доступные через файловые операции, а DTA — системная область для временного хранения данных файловых структур.


section{Каково назначение указателя файла?}

Указатель файла — это внутренняя переменная операционной системы DOS, определяющая текущее положение внутри открытого файла, откуда будет происходить следующая операция чтения или записи.

Когда файл открывается, указатель устанавливается в начало файла (смещение 0). При каждой операции чтения или записи DOS автоматически изменяет указатель — он продвигается на количество считанных или записанных байт.

Однако программист может управлять положением указателя самостоятельно с помощью системной функции 42h (Lseek) прерывания INT 21h. Эта функция позволяет переместить указатель:

относительно начала файла (AL = 0),

относительно текущего положения (AL = 1),

относительно конца файла (AL = 2).

Смещение задаётся в регистрах CX:DX (старшая и младшая части 32-битного значения). После выполнения функция возвращает новое положение указателя в DX:AX.

Назначение указателя — обеспечить произвольный доступ к данным. Благодаря ему можно:

перемещаться к нужным участкам файла;

реализовывать обработку больших файлов;

дозаписывать информацию в конец;

перезаписывать отдельные участки без пересоздания файла.

Таким образом, указатель файла — это аналог курсора в тексте, определяющий текущее место работы с данными. Без него невозможно реализовать гибкую систему ввода-вывода в ассемблерных программах.

\endinput


\printbibliography[title=Список использованных источников] % Автособираемый список литературы

\end{document}