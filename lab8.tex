\chapter{Использование системного сервиса. Работа с файлами}
\begin{itemize}
\item Какие функции системного сервиса прерывания 21h необходимы для работы с
файлами данных?
\item Что такое дескриптор файла?
\item Расшифруйте CON, AUX, PRN и DTA.
\item Каково назначение указателя файла?
\end{itemize}

section{Какие функции системного сервиса прерывания 21h необходимы для работы с файлами данных?}

Работа с файлами в операционной системе MS-DOS осуществляется через вызовы системного прерывания INT 21h, которое предоставляет целый набор функций для открытия, создания, чтения, записи и закрытия файлов. Эти сервисы обеспечивают взаимодействие прикладных программ с файловой системой FAT, скрывая аппаратные детали.

Основные функции работы с файлами:

Создание файла (функция 3Ch)
В регистре AH = 3Ch, в CX задаются атрибуты файла (обычно 0 — обычный файл), а в DS:DX передаётся адрес ASCIIZ-строки с именем файла.
В результате в AX возвращается дескриптор файла — уникальный номер, по которому система идентифицирует открытый файл.

Открытие существующего файла (функция 3Dh)
В AH = 3Dh, в AL — режим доступа (0 = только чтение, 1 = только запись, 2 = чтение/запись), а в DS:DX — адрес имени файла.
Возвращает дескриптор файла в AX.

Закрытие файла (функция 3Eh)
В AH = 3Eh, в BX — дескриптор файла.
Закрывает файл, освобождая ресурсы, связанные с ним.

Чтение из файла (функция 3Fh)
В AH = 3Fh, BX — дескриптор файла, CX — количество байт для чтения, DS:DX — адрес буфера.
После вызова INT 21h в AX возвращается количество реально считанных байт.

Запись в файл (функция 40h)
В AH = 40h, BX — дескриптор файла, CX — количество записываемых байт, DS:DX — адрес данных.
После выполнения возвращается количество записанных байт в AX.

Удаление файла (функция 41h)
В AH = 41h, в DS:DX — адрес имени удаляемого файла.
После вызова файл удаляется из каталога.

Позиционирование (перемещение указателя) в файле (функция 42h)
В AH = 42h, BX — дескриптор файла, AL — способ смещения (0 = от начала, 1 = от текущего положения, 2 = от конца), CX:DX — значение смещения.
После выполнения новый указатель позиции возвращается в DX:AX.

Эти функции позволяют реализовать полный цикл работы с файлами: создание, запись, чтение и закрытие. Все операции с файлами требуют обращения через дескриптор — системный идентификатор, по которому DOS управляет файлами

section{Что такое дескриптор файла?}

Дескриптор файла — это числовой идентификатор, который операционная система MS-DOS присваивает каждому открытому файлу. Этот идентификатор используется программой при выполнении операций чтения, записи или закрытия, вместо непосредственного указания имени файла.

Когда программа вызывает системный сервис INT 21h для открытия или создания файла (функции 3Dh и 3Ch), DOS выделяет для него запись в таблице открытых файлов. В этой записи хранятся:

адрес начала файла на диске;

текущее положение указателя чтения/записи;

атрибуты доступа;

информация о состоянии файла (например, конец файла, ошибки).

Возвращаемое значение — дескриптор (File Handle) — это индекс этой записи. Обычно DOS возвращает маленькое целое число (0, 1, 2 и далее).

Примеры стандартных дескрипторов:

0 — стандартный ввод (клавиатура, CON);

1 — стандартный вывод (экран, CON);

2 — стандартный вывод ошибок (CON).

Таким образом, дескриптор — это средство связи между прикладной программой и внутренними структурами DOS, обеспечивающее эффективную и абстрактную работу с файлами без обращения к физическим адресам носителя.

section{Расшифруйте CON, AUX, PRN и DTA.}

Эти обозначения являются специальными именами устройств DOS, которые система обрабатывает так же, как обычные файлы. Они позволяют обращаться к устройствам ввода-вывода единым способом — через файловые функции (INT 21h).

1. CON (Console)

Обозначает консоль — стандартные устройства ввода и вывода (клавиатура и экран).

Чтение из CON — это ввод с клавиатуры.

Запись в CON — это вывод текста на экран.
Пример: COPY file.txt CON выведет содержимое файла на экран.

2. AUX (Auxiliary)

Представляет дополнительное устройство — чаще всего последовательный порт (COM1).
Используется для связи с модемами и другими последовательными устройствами.

3. PRN (Printer)

Обозначает принтер, обычно подключённый к LPT1.
Запись в PRN приводит к выводу данных на печать.

4. DTA (Disk Transfer Area)

Это область памяти, используемая DOS при операциях с каталогами и файлами.
При вызове функций поиска (например, 4Eh — Find First, 4Fh — Find Next) DOS записывает найденные данные (имя, атрибуты, размер, дату) в DTA.
Адрес DTA можно задать вручную функцией 1Ah прерывания 21h.

Таким образом, CON, AUX, PRN — это логические устройства ввода-вывода, доступные через файловые операции, а DTA — системная область для временного хранения данных файловых структур.


section{Каково назначение указателя файла?}

Указатель файла — это внутренняя переменная операционной системы DOS, определяющая текущее положение внутри открытого файла, откуда будет происходить следующая операция чтения или записи.

Когда файл открывается, указатель устанавливается в начало файла (смещение 0). При каждой операции чтения или записи DOS автоматически изменяет указатель — он продвигается на количество считанных или записанных байт.

Однако программист может управлять положением указателя самостоятельно с помощью системной функции 42h (Lseek) прерывания INT 21h. Эта функция позволяет переместить указатель:

относительно начала файла (AL = 0),

относительно текущего положения (AL = 1),

относительно конца файла (AL = 2).

Смещение задаётся в регистрах CX:DX (старшая и младшая части 32-битного значения). После выполнения функция возвращает новое положение указателя в DX:AX.

Назначение указателя — обеспечить произвольный доступ к данным. Благодаря ему можно:

перемещаться к нужным участкам файла;

реализовывать обработку больших файлов;

дозаписывать информацию в конец;

перезаписывать отдельные участки без пересоздания файла.

Таким образом, указатель файла — это аналог курсора в тексте, определяющий текущее место работы с данными. Без него невозможно реализовать гибкую систему ввода-вывода в ассемблерных программах.

\endinput
