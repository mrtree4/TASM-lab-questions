\chapter{Использование системного сервиса. Ввод данных с клавиатуры. Вывод на экран}
\begin{itemize}
\item Каковы источники и типы прерываний? Каков механизм выполнения прерываний?
\item Опишите алгоритмы преобразования символьных данных в числовые и обратно.
\item Какие существуют функции системного сервиса прерывания Int 21h для организации ввода с клавиатуры и вывода на экран?
\item Как организовать программный вызов системного сервиса?
\end{itemize}

\section{Каковы источники и типы прерываний? Каков механизм выполнения прерываний?}

Прерывание — это механизм, с помощью которого процессор временно приостанавливает выполнение текущей программы и передаёт управление другой программе (обработчику прерывания). После завершения обработки процессор возвращается к месту, где была прервана основная программа.

Источники прерываний делятся на внутренние и внешние, а также на аппаратные и программные.

Основные типы прерываний:

Аппаратные прерывания — инициируются внешними устройствами (например, клавиатурой, таймером, сетевой картой). Когда устройство готово передать данные или требует обслуживания, оно подаёт сигнал на контроллер прерываний, который вызывает соответствующий обработчик.

Программные прерывания — вызываются инструкциями программы. В архитектуре x86 для этого используется команда INT n, где n — номер вектора прерывания (например, INT 21h — вызов системных функций DOS).

Внутренние (исключения) — генерируются самим процессором при возникновении ошибок (например, деление на ноль, переполнение, ошибка защиты памяти).

Механизм выполнения прерывания:

При возникновении прерывания процессор сохраняет контекст — флаги, указатель команд (IP/EIP) и сегмент кода (CS) в стек.

Далее он определяет адрес обработчика прерывания, используя таблицу векторов прерываний, расположенную в нижней области памяти (начиная с адреса 0000:0000).

Процессор переходит по адресу обработчика, загружая из таблицы новые значения CS:IP.

Выполняется программа-обработчик, после чего команда IRET восстанавливает состояние процессора и возвращает управление прерванной программе.

Таким образом, прерывания обеспечивают асинхронное выполнение задач и позволяют эффективно взаимодействовать между процессором, устройствами и операционной системой.

\section{Опишите алгоритмы преобразования символьных данных в числовые и обратно.}

Прерывание — это механизм, с помощью которого процессор временно приостанавливает выполнение текущей программы и передаёт управление другой программе (обработчику прерывания). После завершения обработки процессор возвращается к месту, где была прервана основная программа.

Источники прерываний делятся на внутренние и внешние, а также на аппаратные и программные.

Основные типы прерываний:

Аппаратные прерывания — инициируются внешними устройствами (например, клавиатурой, таймером, сетевой картой). Когда устройство готово передать данные или требует обслуживания, оно подаёт сигнал на контроллер прерываний, который вызывает соответствующий обработчик.

Программные прерывания — вызываются инструкциями программы. В архитектуре x86 для этого используется команда INT n, где n — номер вектора прерывания (например, INT 21h — вызов системных функций DOS).

Внутренние (исключения) — генерируются самим процессором при возникновении ошибок (например, деление на ноль, переполнение, ошибка защиты памяти).

Механизм выполнения прерывания:

При возникновении прерывания процессор сохраняет контекст — флаги, указатель команд (IP/EIP) и сегмент кода (CS) в стек.

Далее он определяет адрес обработчика прерывания, используя таблицу векторов прерываний, расположенную в нижней области памяти (начиная с адреса 0000:0000).

Процессор переходит по адресу обработчика, загружая из таблицы новые значения CS:IP.

Выполняется программа-обработчик, после чего команда IRET восстанавливает состояние процессора и возвращает управление прерванной программе.

Таким образом, прерывания обеспечивают асинхронное выполнение задач и позволяют эффективно взаимодействовать между процессором, устройствами и операционной системой.

\section{Какие существуют функции системного сервиса прерывания Int 21h для организации ввода с клавиатуры и вывода на экран?}

Прерывание INT 21h — это основной интерфейс операционной системы MS-DOS, предоставляющий набор функций (сервисов) для работы с файлами, устройствами, памятью и консолью. Для ввода и вывода символов оно особенно важно, так как позволяет напрямую взаимодействовать с клавиатурой и экраном без обращения к аппаратуре.

Основные функции INT 21h для ввода/вывода:

Функция 01h — Ввод символа с клавиатуры (с эхо)

Ожидает нажатие клавиши.

Отображает введённый символ на экране.

Возвращает ASCII-код символа в регистре AL.
Используется для простого ввода одного символа.

Функция 02h — Вывод символа на экран

Символ для вывода задаётся в регистре DL.

После вызова INT 21h символ отображается на экране.
Это базовый способ вывода символов.

Функция 09h — Вывод строки на экран

В регистре DX задаётся адрес строки, заканчивающейся символом $.

Строка полностью выводится на экран.
Это основной способ вывода текста.

Функция 0Ah — Ввод строки с клавиатуры

В DX передаётся адрес буфера, где будет сохранена введённая строка.

Первый байт буфера задаёт максимальную длину ввода, второй — реальное количество введённых символов.
Используется для ввода текстовых данных с клавиатуры.

Функция 06h — Ввод/вывод без ожидания

Позволяет проверять наличие символов в буфере клавиатуры и выполнять вывод без блокировки.

Таким образом, через прерывание INT 21h реализуется полный набор сервисов для ввода и вывода в текстовых программах DOS, что делает его центральным инструментом взаимодействия между программой и пользователем.

\section{Как организовать программный вызов системного сервиса?}

Программный вызов системного сервиса в среде DOS осуществляется через программное прерывание INT 21h, при этом в определённые регистры загружаются параметры, задающие тип операции и её аргументы.

Общий алгоритм вызова системного сервиса:

В регистр AH помещается номер функции (например, 02h — вывод символа, 09h — вывод строки).

В другие регистры (обычно AL, DL, CX, DX, DS и т. д.) помещаются данные или адреса, необходимые функции.

Выполняется команда:

INT 21h

После выполнения прерывания результаты могут возвращаться в те же регистры (например, в AL возвращается считанный символ).

Пример в теоретическом плане:

чтобы вывести символ, в DL помещают его ASCII-код, в AH — 02h, затем вызывают INT 21h;

чтобы ввести символ, в AH помещают 01h и вызывают INT 21h; символ возвращается в AL.

Таким образом, системные сервисы DOS вызываются программным способом, через инструкцию INT с номером прерывания и передачей параметров через регистры.

Такой механизм обеспечивает стандартный интерфейс между пользовательской программой и операционной системой, независимо от конкретной аппаратной реализации устройств. В этом заключается фундаментальная особенность архитектуры MS-DOS и всей линейки ассемблерных программ для TASM.

\endinput
